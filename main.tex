\documentclass[12pt,openany]{report}
\setlength{\headheight}{52pt}
\usepackage[a4paper,left=1in,right=1in,top=1in,bottom=1in, heightrounded, marginparwidth=2cm, marginparsep=3mm]{geometry}
\usepackage[utf8x]{inputenc}
\usepackage[english]{babel}
\setlength{\parskip}{0.5em}
\usepackage{url}
\usepackage{titlesec}
\setcounter{secnumdepth}{4}
\usepackage{palatino}
\usepackage{tocloft}
\usepackage[nottoc]{tocbibind}
\usepackage{marginnote}
\usepackage{multirow}
\usepackage[authoryear,round]{natbib}
\usepackage{amssymb,amsmath,amsthm,amsfonts}
\usepackage{mathtools}
\usepackage{graphicx}
\graphicspath{{./chapter1/outputs/}}
\usepackage{hyperref}
\usepackage{newpxtext,newpxmath}
\usepackage{fancyhdr}
\usepackage[Conny]{fncychap}
    \ChNameVar{\centering\huge\fontfamily{ppl}\selectfont\rm\bfseries}
    \ChNumVar{\huge\fontfamily{ppl}\selectfont}
    \ChTitleVar{\centering\LARGE\fontfamily{ppl}\selectfont\rm}

\usepackage{xcolor}

\hypersetup{
    colorlinks,
    citecolor=red,
    filecolor=black,
    linkcolor=violet,
    urlcolor=blue
}
\newtheorem{theorem}{Theorem}
\renewcommand\cftchapafterpnum{\vskip5pt}
\renewcommand\cftsecafterpnum{\vskip0pt}

\fancypagestyle{mystyle}{%
  \fancyhf{}% Clear header/footer
  \renewcommand{\headrulewidth}{1pt}% Remove header rule
  \renewcommand{\footrulewidth}{1pt}% Remove footer rule
  \fancyhead[R]{\href{https://github.com/SaiChrisZHANG/empirical-finance-literature}{Github Page}}
  \fancyhead[L]{Empirical Finance: A Review}
  \fancyfoot[L]{\hyperref[ToC-first-page]{To Contents}}
  \fancyfoot[C]{\small \leftmark}
  \fancyfoot[R]{\thepage}
}

\usepackage{minitoc}
\newcommand{\sidenotes}[1]{\marginnote{\raggedright\scriptsize#1}}

\usepackage{paralist}
  \let\itemize\compactitem
  \let\enditemize\endcompactitem
  \let\enumerate\compactenum
  \let\endenumerate\endcompactenum
  \let\description\compactdesc
  \let\enddescription\endcompactdesc
  \pltopsep=1pt
  \plitemsep=1pt
  \plparsep=1pt

\usepackage{etoolbox}
\AtBeginEnvironment{quote}{\vspace{1pt}}
\AtEndEnvironment{quote}{\vspace{1pt}}

\begin{document}

\begin{titlepage}
    \begin{center}
        \vspace*{1cm}
        
        \Huge
        \textbf{Math: From An Economist's Perspective}

        \Large
        \textit{For Personal Reference, happy to circulate}
            
        \vspace{2.5cm}
        
        \LARGE    
        \textbf{Sai Zhang}
            
        \vfill
        
        \large    
        Check the \href{https://github.com/SaiChrisZHANG/Math-in-Economics}{Github Page} for this project, or \href{mailto:saizhang.econ@gmail.com}{email me}!

        \vspace{0.8cm}
        \large
        \today
            
    \end{center}
\end{titlepage}

%%%%%%%% Main content %%%%%%%%%%
%%%%%%%% Chapters are called separately %%%%%%%%%%

\chapter*{Here we go!}

Math is fascinating, certainly. It is clean, organized, beautiful, philosophical,
but it is also hard to grasp. I started this project for one simple purpose: As an Economic Ph.D. student, 
math was not my strongest suit and I NEED to change that. Hence, this
math-learning notebook will be tailored according to the need of an economist,
instead of being, you know, math math.

Here, I cover the math knowledge ranging from basic concepts, to fundamental theories including 
linear algebra and real analysis, and more integrated topics including optimization, dynamic methods stochastic control, etc.
There are several valuable sources I referred to in the process of making this notebook:

First, I thank Prof. Brijesh Pinto at USC Economics for reminding the importance of math and the pleasure of 
playing with math in the math camp prior to my Ph.D. study. Though brief and abstract, the math camp had actually inspired
me to go back to the beginning, really dive in and put together this personal learning notes.

This notebook has two general aspects: math theories and their application in economic research. I organize
the theoretical contents based on \citet{hoy2011mathematics}'s \textit{Mathematics of Economics}, 
\citet{carter2001foundations}'s \textit{Fundations of Mathematical Economics} and \citet{eichhornmathematics}'s 
\textit{Mathematics and Methodology for Economics}; \citet{intriligator2002mathematical}'s \textit{Mathematical Optimization and Economic Theory},
 \citet{vali2014principles}'s \textit{Principles of Mathematical Economics}
and \citet{de2000mathematical}'s \textit{Mathematical Methods and Models for Economists} are my main references
for the application of math thoeries in specific economic questions. Although the above listed books are
rather thorough and well-organized, I also  for some specific topics.


Since this notebook approaches math in an application perspective, I will not only
review the theoretical aspects of each topic, but include modelling simulation techniques
and their application  All the codes related
to this review can be found on \href{https://github.com/SaiChrisZHANG}{my Github page}.

I thank Dr. Svetlana Bryzgalova for her valuable intuitions and impressive
knowledge of the empirical finance literature. Building this review is truly
a memorable journey for me. I would love to share this review and all the related
materials to anyone that finds them useful. And unavoidably, I would make some
typos and other minor mistakes (hopefully not big ones). So I'd really appreciate
any correction. If you find any mistakes, please either set up a branch on Github
or send the mistakes to this email address 
\href{mailto:saizhang.econ@gmail.com}{saizhang.econ@gmail.com}, BIG thanks in advance!

\newpage

\dominitoc
\phantomsection
\label{ToC-first-page}
\tableofcontents

\pagestyle{mystyle}
\chapter{Linear Algebra}
\minitoc

\vspace{0.5cm}
Every investor knows that trading in financial markets is to play
games with time itself. Daily trades determine asset prices at every date and hence
influence the random distribution of future prices as well as the initial
level of pri

\chapter{Real Analysis}

\chapter{Optimization}

\chapter{Dynamic Method}

\newpage
\bibliographystyle{plainnat}
\bibliography{ref.bib}

\end{document}