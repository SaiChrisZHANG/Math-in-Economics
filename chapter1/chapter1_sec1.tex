A \textbf{set} is a collection of elements, normally denoted as
$$
S = \{x (\in A):P(x)\}
$$
where $S$ is the set, $x$ represents elements (add $x\in A$ if $x$ must also belong to set $A$), $P(x)$ represents the property of $x$.

\subsection{Subsets and Set Operations}
\begin{definition}
    If all elements of a set $X$ are also of another set $Y$, then $X$ is a \textbf{subset} of $Y$, which can be written as $X\subseteq Y$.
\end{definition}

After defining subsets, we have two extended definitions: \textbf{proper subset} and \textbf{set equality}.
\begin{definition}
    If all elements of a set $X$ are also of another set $Y$, but not all elements of $Y$ are in $X$, then $X$ is a \textbf{proper subset} of $Y$, written as $X \subset Y$
\end{definition}
\begin{definition}
    If two sets $X$ and $Y$ contain exactly the same elements, $X$ and $Y$ are \textbf{equal}, written as $X=Y$.
\end{definition}

The equality of two sets can also be expressed in another way: $$ X\subseteq Y \text{and} Y\subseteq X \Leftrightarrow X=Y$$.
Now with the basic definitions, we can move to the set operations. Here, all the set operations are based subsets of a \textbf{universal set} $U$.
\begin{definition}\label{def_subset}
    For two subsets of $U$, $X$ and $Y$:
    \begin{enumerate}
        \item[-] Intersection: $X\cap Y=\{x:x\in X \text{ and } x\in Y\}$, if $X$ and $Y$ don't share any common elements, the intersection would be 
        an \textbf{empty/null set} $\varnothing$, $X$ and $Y$ are said to be \textbf{disjoint}
        \item[-] Union: $X\cup Y = \{x:x\in X \text{ or } x\in Y \}$. The union of two sets strictly contain their intersection, or $X\cap Y \subset X\cup Y$
        \item[-] Complement: $X^C(\bar{X}) = \{x\in U:x\notin X\}$. The complement of the universal set $U^C=\varnothing$
        \item[-] Relative difference: $X-Y = {x\in U:x\in X \text{ and } x\notin Y}$.
    \end{enumerate}
\end{definition}

With Definition \ref{def_subset} in mind, we have:
\begin{enumerate}
    \item[-] $(X\cap Y)^C = X^C\cup Y^C$, $(X\cup Y)^C = X^C\cap Y^C$, $X-Y = X\cap Y^C$, $X\subseteq Y \rightarrow Y^C \subseteq X^C$
    \item[-] $X\subseteq Y \rightarrow X\cup(Y-X)=Y$, $X-Y \subseteq X\cup Y$, $X\cap Y =\varnothing\rightarrow Y\cap X^C=Y$
    \item[-] $X-Y = X-(X\cap Y)=(X\cup Y)-Y$, $(X-Y)-Z=X-(Y\cup Z)$, $X-(Y-Z) = (X-Y)\cup(X\cap Z)$, $(X\cup Y)-Z = (X-Z)\cup(Y-Z)$, $X-(Y\cup Z)=(X-Y)\cap(X-Z)$
\end{enumerate}

Definition \ref{def_subset} can be extended to collections of sets.
\begin{definition}
For a collection of sets $\mathcal{S}$:
\begin{enumerate}
    \item[-] Union: $\bigcup_{S\in\mathcal{S}}S=\{x:x\in S \text{ for some } S\in \mathcal{S}\}$
    \item[-] Intersection: $\bigcup_{S\in\mathcal{S}}S=\{x:x\in S \text{ for every } S\in \mathcal{S}\}$
\end{enumerate}
\end{definition}

Apart from union, intersection, complement and relative difference (subtract), we can also take the \textbf{product} of two sets:
\begin{definition}\label{def_cartesian_product}
    The \textbf{Cartesian product} of two sets $X$ and $Y$ is the set of \textbf{ordered} pairs $X\times Y = \{(x,y):x\in X, y\in Y\}$
\end{definition}
Notice that \textbf{order} is important, meaning that $(x,y)$ is different from $(y,x)$. Equally, we can generalize this definition as 
$\prod^n_{k=1}S_k=\{(s_1,\cdots,s_n)|s_1\in S_1,\cdots,s_n\in S_n\}$. If we want to denote the product of all BUT the $i$th set, we can use:
$X_{-i}=X_1\times \cdots \times X_{i-1}\times X_{i+1}\times\cdots\times X_n$.

Another two definitions focus on two important collections of subsets:
\begin{definition}\label{def_collect_subset}
    \begin{enumerate}
        \item[-] Partition: a collection of \textbf{disjoint} subsets of $U$, of which the union is $U$, 
        i.e., for $U$, if the collection of its $n$ subsets $\{X_i\},i=1,\cdots,n$ is a partition of $U$, 
        $\{X_i\}$ must satisfy: $X_i\cap X_j=\varnothing, \forall i,j=1,\cdots,n,i\neq j$ and $\bigcup^n_{i=1} X_i=U$. On element level,
        each element of $U$ lies in one and ONLY one subset of the partition.
        \item[-] Power set: the power set $\mathcal{P}(X)$ of a set $X$ is the set of ALL its subsets, 
        i.e., $\mathcal{P}(X)\{A:A\subseteq X \}$\footnote{Fact: For any set $X$ with a finite number $n$ of elements, its power set $\mathcal{P}(X)$ has $2^n$ elements, including one $\varnothing$. This can be proved as: $\binom{n}{0}+\binom{n}{1}+\binom{n}{2}+\cdots+\binom{n}{n}=(x+y)^n \mid_{x=1,y=1}=2^n$}
    \end{enumerate}
\end{definition}

\subsection{Relations and Order}
For two sets $X$ and $Y$, any subset $R$ of their product $X\times Y$ is called a \textit{binary relation}, any pair of elements $(x,y)\in R\subseteq X\times Y$ can be described as $x$ is \textbf{related} to $y$, denoted as $x\mathbf{R}y$. A relation is a subset of the product, expressing the relationship between elements.