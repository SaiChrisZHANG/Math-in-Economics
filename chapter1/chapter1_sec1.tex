A \textbf{set} is a collection of elements, normally denoted as
$$
S = \{x (\in A):P(x)\}
$$
where $S$ is the set, $x$ represents elements (add $x\in A$ if $x$ must also belong to set $A$), $P(x)$ represents the property of $x$.

\subsection{Subsets and Set Operations}
\begin{definition}
    If all elements of a set $X$ are also of another set $Y$, then $X$ is a \textbf{subset} of $Y$, which can be written as $X\subseteq Y$.
\end{definition}

After defining subsets, we have two extended definitions: \textbf{proper subset} and \textbf{set equality}.
\begin{definition}
    If all elements of a set $X$ are also of another set $Y$, but not all elements of $Y$ are in $X$, then $X$ is a \textbf{proper subset} of $Y$, written as $X \subset Y$
\end{definition}
\begin{definition}
    If two sets $X$ and $Y$ contain exactly the same elements, $X$ and $Y$ are \textbf{equal}, written as $X=Y$.
\end{definition}

The equality of two sets can also be expressed in another way: $$ X\subseteq Y \text{and} Y\subseteq X \Leftrightarrow X=Y$$.
Now with the basic definitions, we can move to the set operations. Here, all the set operations are based subsets of a \textbf{universal set} $U$.
\begin{definition}
    For two subsets of $U$, $X$ and $Y$:
    \begin{enumerate}
        \item[-] Intersection: $X\cap Y=\{x:x\in X \text{ and } x\in Y\}$, if $X$ and $Y$ don't share any common elements, the intersection would be 
        an \textbf{empty/null set} $\varnothing$, $X$ and $Y$ are said to be \textbf{disjoint}
        \item[-] Union: $X\cup Y = \{x:x\in X \text{ or } x\in Y \}$. The union of two sets strictly contain their intersection, or $X\cap Y \subset X\cup Y$
        \item[-] Complement: 
    \end{enumerate}
\end{definition}